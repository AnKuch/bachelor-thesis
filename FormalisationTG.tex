% !TEX root = Bachelorarbeit.tex
\section{Formalisation of the Take-Grant Model}\label{sec:Formalisation}
\subsection{Capabilities}
In the Take-Grant model for seL4 \cite{TakeG}, where I got the formalisation from, the authors waived the usual differentation between subjects and objects and called all kernel objects \textit{entities}. \\ \\
The entities memory address identifies them and is modeled as a natural number. \\ \\
	{\relsize{-1}
	\textbf{type$\_$synonym} \texttt{ entity$\_$id = nat}} \\ \\
With each capability a set of rights is associated. There are four access rights in the system model:	\\ \\
	{\relsize{-1}
	\textbf{datatype} \texttt{ rights = Read | Write | Grant | Create}} 
	
\begin{itemize}	
\item \textit{Read} authorises the reading of information from another entity. 
\item \textit{Write} authorises the writing of information to another entity. 
\item \textit{Grant} authorises the passing of a capability to another entity. 
\item \textit{Create} authorises the creation of new entities, which models the behavior of untyped memory objects. 
\end{itemize} 
	
A capability has two fields:
\begin{enumerate}
\item An identifier which names a target-entity
\item A set of rights which defines which system operations the source-entity is authorized to perform on the target-entity. 
\end{enumerate} 

	{\relsize{-1}
	\textbf{record} 
	\texttt{	
	\begin{tabular}[t]{ll}
	cap = & entity :: entity$\_$id \\
	      & rights :: rights set
	\end{tabular}} }
	\\ \\ \\
	
An entity has a set of capabilities: \\ \\
	{\relsize{-1}
	\textbf{record } \texttt{ 						
	entity = caps :: cap set}} \\ \\
	
The systems state includes two flields: 
\begin{enumerate}
\item The \texttt{heap}, which stores the entities of the system like an array form address \texttt{0} up to and excluding \texttt{next$\_$id}.
\item \texttt{next$\_$id} contains slot for the next entity without overlapping with an existing one. 
\end{enumerate} 
	{\relsize{-1}	
	\textbf{record}
	\texttt{
	\begin{tabular}[t]{ll}
	state =	& heap :: entity$\_$id $\Rightarrow$ entity \\
			& next$\_$id :: entity$\_$id
	\end{tabular}}}  \\
	
\subsection{System Operations}\label{sec:System Operations}
The data type \texttt{sysOps} defines the different system operations of the seL4. \\ \\
	{\relsize{-1}	
	\textbf{datatype}
	\texttt{
	\begin{tabular}[t]{lll}
	sysOps	&	=	&	SysNoOp entity$\_$id \\
			&	|	&	SysRead	entity$\_$id cap \\
			&	|	&	SysWrite entity$\_$id cap \\
			&	|	&	SysCreate entity$\_$id cap cap \\
			&	|	&	SysGrant entity$\_$id cap cap rights set \\
			&	|	&	SysRemove	entity$\_$id cap cap \\
			&	|	&	SysRevoke entity$\_$id cap
	\end{tabular}}} \\ \\ 
	
The entity$\_$id in each operation is the entity initiating the operation. The first named capability is the one that is being invoked. The second capability for \texttt{SysCreate} points to the target entity for the new capability. For \texttt{SysGrant} it is the passed capability and for \texttt{SysRemove} it is the one that has to be removed. The rights set in \texttt{SysGrant} is necessary for the initiating entity to have the option only to transport a subset of the authority it offers to the receiver. 	\\

The \texttt{diminish} function applies this mask on the given acces rights: \\ \\
	{\relsize{-1}	
	\texttt{
	diminish :: "cap $\Rightarrow$ rights set $\Rightarrow$ cap" \textbf{where} \\
	diminish c R $\equiv$ c$\llparenthesis$rights := rights c $\cap$ \textit{R}$\rrparenthesis$}} \\ \\ \\
	
\texttt{legal} defines on what terms any system operation is allowed. \\ \\
	{\relsize{-1}
	\texttt{legal :: "sysOPs $\Rightarrow$ state $\Rightarrow$ bool" \texttt{where}}\\ \\
	
	\texttt{
	\begin{tabular}{lllll}
	  	&	"legal	&	(SysNoOp e) s	&	=	&	isEntityOf s e" \\
	|	& 	"legal	&	(SysCreate e c$_1$ c$_2$) s	&  =	& (isEntityOf s e $\wedge$ {c$_1$, c$_2$} $\subseteq$ caps$\_$of s e $\wedge$ \\ & & & & Grant $\in$ rights c$_2$ $\wedge$ Create $\in$ rights c$_2$)" \\
	| 	& "legal 	& 	(SysRead e c) s	&	=	&	(isEntityOf s e $\wedge$ c $\in$ caps$\_$of s e $\wedge$ Read \\ & & & & $\in$ rights c)" \\
	|	&	"legal 	&	(SysWrite e c) s&	= 	&	(isEntityOf s e $\wedge$ c $\in$ caps$\_$of s e $\wedge$ Write \\ & & & & $\in$ rights c)" \\
	| 	&	"legal 	&	(SysGrant e c$_1$ c$_2$ r) s & = & (isEntityOf s e $\wedge$  isEntityOf s (entity c$_1$) \\ & & & & $\wedge$ {c$_1$,c$_2$} $\subseteq$ caps$\_$of s e $\wedge$ Grant $\in$ rights c$_1$)" \\
	| 	&	"legal 	&	(SysRemove e c$_1$ c$_2$) s	&  =	& (isEntityOf s e $\wedge$ c$_1$ $\in$ caps$\_$of s e)" \\
	|	&	"legal	&	(SysRevoke e c) s	&	=	&	isEntityOf s e $\wedge$ c $\in$ caps$\_$of s e"
	\end{tabular}}} \\ \\ \\
	
The function \texttt{isEntityOf} tests the existence of an \texttt{entity$\_$id}. \texttt{Caps$\_$of} issues the set of all capabilities  contained in the entity with the address \texttt{r} in state \texttt{s}. \\ \\
The original executions of \texttt{SysRead} and \texttt{SysWrite} don't have an underlying function. For implying the noninterference property I have to include what happens if an entity reads or writes a value from another entity. For this purpose I defined a \texttt{readOperation} and a \texttt{writeOperation}. \\
	The \texttt{step'} and \texttt{step} functions define the execution of a single system operation: \\ \\
	{\relsize{-1}	
	\texttt{step' :: "sysOPs $\Rightarrow$ state $\Rightarrow$ state" \textbf{where}} \\
	\texttt{
	\begin{tabular}{lllll}
		&	"step'	&	(SysNoOp e) s	&	=	&	s" \\
	|	&	"step'	&	(SysRead e c) s	&	=	&	readOperation e c s" \\
	|	&	"step'	&	(SysWrite e c) s &	=	&	writeOperation e c s" \\
	|	&	"step'	&	(SysCreat e c$_1$ c$_2$) s & = & createOperation e c$_1$ c$_2$ s" \\
	|	&	"step'	&	(SysGrant e c$_1$ c$_2$ \textit{R}) s & = & grantOperation e c$_1$ c$_2$ \textit{R} s" \\
	|	&	"step'	&	(SysRemove e c$_1$ c$_2$) s & = & removeOperation e c$_1$ c$_2$ s" \\
	|	&	"step'	&	(SysRevoke	e c) s & =	&	revokeOperation e c s" 
	\end{tabular}} \\ \\
	\texttt{step :: "sysOps $\Rightarrow$ state $\Rightarrow$ state" \textbf{where} \\
	step cmd s $\equiv$ if legal cmd s then step' cmd s else s}} \\ \\
	The new defined functions \texttt{readOperation} and \texttt{writeOperation}: \\ \\
	{\relsize{-1}	
	\texttt{readOperation :: "entity$\_$id $\Rightarrow$ cap $\Rightarrow$ modify$\_$state" \textbf{where}} \\
	\texttt{"readOperation e c s $\equiv$ s$\llparenthesis$ heap := (heap s)(e := $\llparenthesis$caps = caps$\_$of s e, eValue = value$\_$of s (entity c)$\rrparenthesis$)$\rrparenthesis$"}}\\ \\
	{\relsize{-1}	
	\texttt{writeOperation :: "entity$\_$id $\Rightarrow$ cap $\Rightarrow$ modify$\_$state" \textbf{where}} \\
	\texttt{"writeOperation e c s $\equiv$ s$\llparenthesis$ heap := (heap s)(entity c := $\llparenthesis$caps = caps$\_$of s (entity c), eValue = value$\_$of s e|))|)"}}\\ \\
	The rest of the system operations stay as they are: \\ \\
	{\relsize{-1}
	\texttt{createOperation :: "entity$\_$id $\Rightarrow$ cap $\Rightarrow$ cap $\Rightarrow$ modify$\_$state" \textbf{where}} \\
  \texttt{createOperation e c$_1$ c$_2$ s $\equiv$ \\
  \begin{tabular}{ll}
            let & nullEntity = $\llparenthesis$cap = {}, eValue = NULL$\rrparenthesis$ ; \\
                & newCap = $\llparenthesis$entity = next$\_$id s, rights = all$\_$rights$\rrparenthesis$; \\
                & newTarget = $\llparenthesis$caps = {newCap} ∪ caps$\_$of s (entity c$_2$), eValue = NULL$\rrparenthesis$ \\
             in & s$\llparenthesis$heap := (heap s)(entity c$_2$ := newTarget, next$\_$id s := nullEntity), next$\_$id := next$\_$id s+1$\rrparenthesis$"
   \end{tabular} \\ \\ \\
    grantOperation :: "entity$\_$id $\Rightarrow$ cap $\Rightarrow$ cap $\Rightarrow$ rights set $\Rightarrow$ modify$\_$state" \textbf{where} \\
  "grantOperation e c$_1$ c$_2$ \textit{R} s $\equiv$ \\
  s$\llparenthesis$heap := (heap s)(entity c$_1$ := $\llparenthesis$caps = {diminish c$_2$ \textit{R}} $\cup$ caps$\_$of s (entity c$_1$), eValue = value$\_$of s \\(entity c$_1$)$\rrparenthesis$)$\rrparenthesis$"\\ \\ \\
   removeOperation :: "entity$\_$id $\Rightarrow$ cap $\Rightarrow$ cap $\Rightarrow$ modify$\_$state" \textbf{where} \\
  "removeOperation c$_1$ c$_2$ s $\equiv$ s$\llparenthesis$heap := (heap s)(entity c$_1$ := $\llparenthesis$caps = caps$\_$of s (entity c$_1$) - {c$_2$}, eValue = value$\_$of s (entity c$_1$)$\rrparenthesis$)$\rrparenthesis$"}}
% !TEX root = Bachelorarbeit.tex
\chapter{Validation of Noninterference}\label{ValNon}
After the formalisations in Chapter \ref{sec:Formalisation} and \ref{FormNon} I try to establish noninterference for the different system operations.\\ \\
First I tried to identify the conditions under which noninterference holds. For this I constructed a small model of the system with one Low-level-Subsystem and one High-level-Subsystem, with entities in them and tested for different right-sets and different operations if the noninterference-property holds. \\
The following displays this approach for the \texttt{write} operation from a low level domain on a high level domain. \\
I assume: \\ 
\begin{itemize}
\item H denotes a High level domain that implements the subsystem 'H'
\item L denotes a Low level domain that implements the subsystem 'L'
\item e$_1$ is an entity in H and e$_2$ is an entity in L
\item s $\overset{\text{L}}{\sim}$ t
\end{itemize} 
\begin{figure}[H]
\pgfuseimage{WriteGraphic1}
\caption{Noninterfernce of Write 1}
\end{figure}
To check noninterference I examine if the criterias for \texttt{aquiv$\_$nonin s' t' L} are fulfilled after the execution of the \texttt{write operation} and the named preconditions. \\ 
The \texttt{write operation} in the extended model satisfies the noninterference property: 
\begin{itemize}
\item We have \texttt{value$\_$of s' e = value$\_$of s e $\wedge$ caps$\_$of s' e = caps$\_$of s e} and \texttt{subSys s' e = subSys s e} as the \texttt{write operation} on the entity, \texttt{c$_1 \in e_2$} is pointing on, changes an entity e$_1 \in$ H and does not affect an entity $\in$ L.
\item We have \texttt{value$\_$of s e = value$\_$of t e $\wedge$ caps$\_$of s e = caps$\_$of t e} and \texttt{subSys s e = subSys t e} as one of the preconditions was s $\overset{\text{L}}{\sim}$ t. I defined the equivalence relation with the function aquiv$\_$nonin s t L, which is equal to the requirement. 
\item The \texttt{step} function first checks whether the execution of the system operation is legal, if not the new state t' equals the old state t. \\
\texttt{value$\_$of t e = value$\_$of t' e $\wedge$ caps$\_$of t e = caps$\_$of t' e} and \texttt{subSys t e = subSys t' e} as \texttt{write} is not part of c$_1$. So legal(SysWrite e$_2$ c$_1$) s = false what leads to t=t'. \\
\end{itemize} 
\textbf{In the following cases the proof looks always the same. So I shorten it:} \\ \\
\textbf{Preconditions:} \\ 
\begin{tabular}{ll}
* & s $\overset{\text{L}}{\sim}$ t $\Rightarrow$ aquiv$\_$nonin s t L	\\ 
** & writeOperation e$_2$ c$_1$  changes e$_1$ $\in$ H no e $\in$ L \\
*** & legal(SysWrite e$_2$ c$_1$) t = false $\Rightarrow$ t=t'
\end{tabular} \\ \\ 
\textbf{Proof of the noninterference property for Write 1:} \\ 
For all e$\in$L, we have: \\
\begin{tabular}{ll}
& (value$\_$of s' e $\overset{\text{**}}{=}$ value$\_$of s e $\overset{\text{*}}{=}$ value$\_$of t e $\overset{\text{***}}{=}$ value$\_$of t' e \\
$\wedge$ & caps$\_$of s' e $\overset{\text{**}}{=}$ caps$\_$of s e $\overset{\text{*}}{=}$ caps$\_$of t e $\overset{\text{***}}{=}$ caps$\_$of t' e \\
$\wedge$ & subSys s' e $\overset{\text{**}}{=}$ subSys s e $\overset{\text{*}}{=}$ subSys t e $\overset{\text{***}}{=}$ subSys t' e)
\end{tabular} \\
Hence aquiv$\_$nonin s' t' L , what is equivalent to s' $\overset{\text{L}}{\sim}$ t' \\ \\
\textbf{With s' $\overset{\text{L}}{\sim}$ t' the noninterference property for \texttt{write} is fulfilled.} 
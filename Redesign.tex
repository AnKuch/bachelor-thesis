% !TEX root = Bachelorarbeit.tex
\chapter{Redesign of the Take-Grant-Model}\label{Redesign}
This procedure worked until I came to the remove-operation. There I noticed the issue, that an entity in the given model is allowed to delete a capability and with that also an object in another domain without any restrictions:
\begin{figure}[H]
\pgfuseimage{RemoveGraphic1}
\caption{No confidentiality for Remove}
\end{figure}
In this example the L domain knows that the \texttt{removeOperation} was performed in the H domain as the capability c$_2$ was deleted. As a consequence the noninterference property is not achieved. \\
To study that problem I decided to classify the entities by their types, corresponding to the kernel specification \cite{Manual}:
\begin{itemize}
\item Untyped 
\item TCB
\item Synchronous IPC Endpoint (SEP)
\item Asychronous IPC Endpoint (AEP)
\item CNode
\item VSpace
\item Interrupt Controller 
\item Interrupt Handler
\item Shared Pages (Pages (Frames) can be shared between domains. The corresponding capability has to be copied and then mapped in the VSpace of the other domain.)
\end{itemize}
The following table shows the different object types with the different operation executable on them and the corresponding take-grant system calls: 
\begin{table}[H]
\begin{tabular}{|l|c|c|}
\hline
Capability Type & Concrete Kernel & Take-Grant-Model \\
\hline
\hline
Untyped & Retype & sequence of \textit{SysCreate} \\
& Revoke & \textit{SysRevoke} \\
\hline
TCB & ThreadControl & \textit{SysNoOP, SysGrant} \\
& Exchange Registers & \textit{SystWrite} or \textit{SysRead} \\
& Yield & \textit{SysNoOP} \\
\hline
Synchronous IPC & Send IPC & \textit{SysWrite} or \textit{SysNoOP} \\
(Endpoint) & Wait IPC & \textit{SysRead} \\
& Grant IPC & \textit{SysWrite, SysGrant} or \textit{SysNoOP} \\
\hline
Asynchronous IPC & Send Event & \textit{SysWrite} \\
(AsyncEndpoint) & Wait Event & \textit{SysRead} \\
\hline
CNode & imitate & \textit{SysGrant} \\
& mint & \textit{SysGrant} \\
& Remove & \textit{SysRemove} \\
& Revoke & \textit{SysRevoke} \\
& Move & \textit{SysGrant, SysRemove} \\
& Recycle & \textit{SysRevoke}, sequence of \textit{SysRemove} \\
\hline
VSpace & Install Mapping & \textit{SysGrant} \\
& Remove Mapping & \textit{SysRemove} \\
& Remap & \textit{SysRemove, SysGrant} \\
& initialise & \textit{SysNoOP} \\
\hline
InterruptController & Register interrupt & \textit{SysGrant} \\
& Unregister interrupt &  \textit{SysRemove}\\
\hline
Interrupt Handler & Acknowledge interrupt & \textit{SysWrite}\\
\hline
\end{tabular}
\caption{Relationship: operation of concrete kernel $\longleftrightarrow$ of protection model \cite{PhDseL4}} \end{table}
To discern the different object types I need to revise the entity record and the preconditions for the different system operations. \\ \\ 
New dataype for the object types: \\
{\relsize{-1}
	\textbf{datatype} 
	\texttt{
	\begin{tabular}[t]{lll}
	eType & = & Untyped \\
			 & | & TCB \\
			 & | & SEP \\
			 & | & AEP \\
			 & | & SPage \\
			 & | & CNode \\
			 & | & VSpace \\
			 & | & IContr \\
			 & | & IHandl							
	\end{tabular}}} \\ 
The final version of the \texttt{entity} record: \\
	{\relsize{-1}
	\textbf{record} 
	\texttt{
	\begin{tabular}[t]{ll}
	entity = & caps :: cap set \\
			 & eValue :: nat \\
			 & eType :: eType										
	\end{tabular}}} \\ \\
	The revised version of the \texttt{legal} function:\\
		{\relsize{-1}
	\texttt{legal :: "sysOPs $\Rightarrow$ state $\Rightarrow$ bool" \texttt{where}}\\ \\
	\texttt{
	\begin{tabular}{lllll}
	  	&	"legal	&	(SysNoOp e) s	&	=	&	isEntityOf s e" \\
	|	& 	"legal	&	(SysCreate e c$_1$ c$_2$) s	&  =	& (isEntityOf s e $\wedge$ {c$_1$, c$_2$} $\subseteq$ caps$\_$of s e $\wedge$ \\ & & & & Grant $\in$ rights c$_2$ $\wedge$ Create $\in$ rights c$_2$) $\wedge$ \\ & & & & eType (entity c$_1$ = Untyped" \\
	| 	& "legal 	& 	(SysRead e c) s	&	=	&	(isEntityOf s e $\wedge$ c $\in$ caps$\_$of s e $\wedge$ Read \\ & & & & $\in$ rights c) $\wedge$ eType (entity c) = TCB $\vee$ SEP $\vee$ \\ & & & & AEP $\vee$ SPage" \\
	|	&	"legal 	&	(SysWrite e c) s&	= 	&	(isEntityOf s e $\wedge$ c $\in$ caps$\_$of s e $\wedge$ Write \\ & & & & $\in$ rights c) $\wedge$ eType  (entity c) = TCB $\vee$ SEP $\vee$ \\ & & & & AEP $\vee$ IHandl $\vee$ SPage" \\
	| 	&	"legal 	&	(SysGrant e c$_1$ c$_2$ r) s & = & (isEntityOf s e $\wedge$  isEntityOf s (entity c$_1$) \\ & & & & $\wedge$ {c$_1$,c$_2$} $\subseteq$ caps$\_$of s e $\wedge$ Grant $\in$ rights c$_1$) $\wedge$ \\ & & & & eType (entity c$_1$)  = TCB $\vee$ SEP $\vee$ CNode $\vee$ VSpace \\ & & & & $\vee$ IContr" \\
	| 	&	"legal 	&	(SysRemove e c$_1$ c$_2$) s	&  =	& (isEntityOf s e $\wedge$ c$_1$ $\in$ caps$\_$of s e)  $\wedge$ \\ & & & & eType (entity c$_1$) = CNode $\vee$ VSpace $\vee$ IContr" \\
	|	&	"legal	&	(SysRevoke e c) s	&	=	&	isEntityOf s e $\wedge$ c $\in$ caps$\_$of s e  $\wedge$ \\ & & & & eType (entity c) = Untyped $\vee$ CNode"
	\end{tabular}}} \\ \\ \\
As mentioned in chapter \ref{sec:System Operations} (System Operations) the step function first proves whether a system operation is "legal" in state s. If it is, the system operation is performed. Otherwise the new state \texttt{s'} is defined as \texttt{s' = s}. This means that if a system operation is not legal nothing happens. 
For the validation I took a subsystem (SS1) of one domain (D1) and another subsystem (SS2) of a second domain (D2). \\
In chapter \ref{sec:KernelObjects} (Kernel Objects) I explained that the only communication between domains goes through \textit{Asynchronous Endpoints} and \textit{Shared Pages}. \\
Figure \ref{overview} pictures an example of how the objects and methods can be placed in the domains and how the connection to \textit{Asynchronous Endpoints} and \textit{Shared Pages} is implemented if the information is allowed to flow from domain 1 to domain 2: D1$\leadsto$D2 but D2$\not\leadsto$D1. \\ 
The connections between the objects inside a domain are just examples. Exept from those from each TCB to a VSpace and a CSpace. At boot time the initial resourcemanager gives authorities to the applications. If information is allowed to flow from domain 1 to domain 2, domain 1 gets write rights on the shared SPage and AEP object and domain 2 only read on both of them. 
\begin{flushleft}
\begin{figure}[H]
\pgfuseimage{OverviewObjects2}
\caption{Objects and methods in the kernel}
\label{overview}
\end{figure}
\end{flushleft}
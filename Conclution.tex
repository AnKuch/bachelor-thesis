% !TEX root = Bachelorarbeit.tex
\chapter{Conclusion}
Summarized I tried to show the noninterference property on the take-grant model as it was specified by the team of NICTA in the paper \href{http://ts.data61.csiro.au/publications/nicta_full_text/1474.pdf}{%
		"Verified Protection Model of the seL4 Microkernel"}\cite{TakeG}. This attempt failed so I had to make the model more precise based on my alalysis in Chapter \ref{ValNon} of noninterference in the original model. That means I defined \texttt{read} and \texttt{write} operations, a value and object type for entities and a check if the object type is able to perform the particular system operation. \\
		\textit{Theorem 1} in Chapter \ref{ValNon} formed the theory of this thesis. 
		With the adaptions in Chapter \ref{Redesign} it was feasible to investigate \textit{Theorem 1} is fulfilled for the system operations of the take-grant model. \\ \\
		The conclusion of the thesis is that the original model is not appropriate to show noninterference on it. With the extended one it was possible and every system operation fulfills it. \\ \\
		As a next step the noninterference property should be specified and verified formally for the extended model. This can also be done with the theorem proof assistant Isabelle/HOL.  
% !TEX root = Bachelorarbeit.tex
\subsection{Noninterference}
Noninterference is an enhancement of the information flow model, first published by Goguen and Meseguer in 1982. It ensures that objects and subject from different security levels do not interfere with those at other levels. In the model variables are classified to be L (low security) or H (high security, private) variables. The goal is to prevent information to flow from H variables to L variables. \\
I use the noninterference formulation as it is used in Geoffrey Smiths \href{http://users.cis.fiu.edu/~smithg/papers/sif06.pdf}{%
		Principles of Secure Information Flow Analysis}. \cite{InfFlow}, which reads "Program c satisfies noninterference if, for any memories $\mu$ and $\nu$ that agree on L variables, the memories produced by running c on $\mu$ and on $\nu$ also agree on L variables (provided that both runs terminate successfully)." \\
This means that, if in a program two states are equivalent on a low level domain, then they are still equivalent on this level after a program was executed.\\ As they are equivalent, also if the program was not executed in one of the states, this implies that the low level domain not only is not able to get the information of the program but even can not recognize if the program was executed. The execution of the program in a high level domain has no impact on the low level domain. This implies that no information flows from the high level to the low level domain. \\
Central to noninterference is the notion of a \textit{policy} $\leadsto$. It specifies the allowed information  flows between domain. $L \leadsto H$ if information is allowed to flow from domain L to domain H. \\
The model says two memories $\mu$ and $\nu$ agree on L variables if they fullfil an equivalence relation $\mu$ $\overset{\text{L}}{\sim}$ $\nu$. \\
The exact formalisation of noninterference for the validation follows in Chapter \ref{FormNon}.